\documentclass[11pt]{article}
% motion_header.tex
% include this before the \begin{document} part of your LaTeX source file
%
\usepackage{fourier,ifthen,xspace,amsmath,array}
\usepackage[svgnames, dvipsnames]{xcolor}
\usepackage{graphicx}
\usepackage{siunitx}
\usepackage{xkeyval,calc,moreverb}
\usepackage[breakable,skins]{tcolorbox}
\usepackage{booktabs}
\usepackage{paralist}

% booleans we need
\newboolean{bw}
\setboolean{bw}{false}
\newboolean{solutions}

% counters

\newcounter{problemctr}

% colors

\definecolor{probLineColor}{rgb}{0.2,0.6,0.2}
\definecolor{probLabelColor}{rgb}{0,0,0} % 0.2,0.2,0.7}
\definecolor{probStatementColor}{rgb}{0,0,0}
\definecolor{solColor}{rgb}{0.2,0.6,0.2}

% lengths

\newlength{\probLineWidth} \setlength{\probLineWidth}{1.25pt}
\newlength{\solLineWidth} \setlength{\solLineWidth}{0.5pt}

% oops!
\setlength{\probLineWidth}{0pt}
\setlength{\solLineWidth}{0pt}

% commands

\newcommand{\probdir}{/Users/saeta/Documents/Courses/p24/motion/prob/}
\newcommand{\figdir}{/Users/saeta/Documents/Courses/p24/motion/figs/}
\renewcommand{\probdir}{}
\renewcommand{\figdir}{}
\newcommand{\prob}[2]{% arguments are chapter then problem
  \input{\probdir ch#1/#1P#2}
}

\newcommand{\fig}[2][2]{%
 \vspace*{-.1in}
  \begin{center}%
    \includegraphics[width=#1in]{\figdir#2}
  \end{center}%
}

\newcommand{\SolutionHead}{Solution:}
\newcommand{\theexercise}{}
\newcommand{\PNScolor}[1]{\ifthenelse{\boolean{bw}}{}{\color{#1}}}
\newcommand{\SOL}[1]{\ifthenelse{\equal{#1}{0}}%
  {
\let\solution=\comment
\let\endsolution=\endcomment
\setboolean{solutions}{false}
}%
{\setboolean{solutions}{true}%
\renewenvironment{solution}%
{\par\noindent{\color{solColor}\rule{\linewidth}{\solLineWidth}%
\ifthenelse{\equal{#1}{}}%
{}%
{\par\smallskip\noindent\textbf{\SolutionHead}}}}{}}
}


\newcommand{\val}[2]{\ensuremath{#1~\mathrm{#2}}\xspace}	% number plus unit
\newcommand{\pval}[2]{\ensuremath{\left(#1~\mathrm{#2}}\right)\xspace}
\newcommand{\hval}[2]{\ensuremath{#1$-$\mathrm{#2}}\xspace}
\newcommand{\dg}[1]{\ensuremath{#1^{\circ}}\xspace}
\newcommand{\half}{\ensuremath{\frac{1}{2}}\xspace}
\newcommand{\ux}{\ensuremath{\mathbf{\hat{x}}}\xspace}
\newcommand{\uy}{\ensuremath{\mathbf{\hat{y}}}\xspace}
\newcommand{\uz}{\ensuremath{\mathbf{\hat{z}}}\xspace}
\newcommand{\ur}{\ensuremath{\mathbf{\hat{r}}}\xspace}
\newcommand{\vv}{\ensuremath{v}\xspace}

\newcommand{\xaxis}{$x$ axis\xspace}
\newcommand{\yaxis}{$y$ axis\xspace}
\newcommand{\zaxis}{$z$ axis\xspace}
\newcommand{\xyplane}{$xy$ plane\xspace}
\newcommand{\xzplane}{$xz$ plane\xspace}
\newcommand{\DOT}{\ensuremath{\boldsymbol{\cdot}}}

\newcommand{\bo}{\ensuremath{\boldsymbol{\omega}}\xspace}
\newcommand{\bO}{\ensuremath{\boldsymbol{\Omega}}\xspace}
\newcommand{\VEC}[1]{\ensuremath{\mathbf{#1}}\xspace}
\newcommand{\AU}{\ensuremath{\mathrm{A.U.}}\xspace}
\newcommand{\DB}{\ensuremath{\mathrm{dB}}\xspace}
\newcommand{\degC}[1]{\ensuremath{#1^{\circ}\mathrm{C}}\xspace}


% environments
\newcommand{\problabel}{Problem}
\newenvironment{problem}[1][]%
{
  \begingroup\refstepcounter{problemctr}
  \ifthenelse{\boolean{solutions}}%
    {
      {\vspace*{-10pt}\noindent\PNScolor{probLineColor}%
        \rule[-5pt]{\linewidth}{\probLineWidth}}
    }{}
  \begin{list}{%
      \PNScolor{probLabelColor}\textbf{\problabel\theexercise}%
    }{%
      \setlength{\labelsep}{0pt}%
      \setlength{\leftmargin}{0in}%
      \setlength{\labelwidth}{0pt}%
      \setlength{\listparindent}{0pt}
    }%
    \PNScolor{probStatementColor}%
	\item%
	\ifthenelse{\equal{#1}{}}%
      {}%
      {\textbf{ -- #1}}%
      \quad
}%
{%
  \end{list}\endgroup%
  \PNScolor{black}%
}

\newcommand{\PAGE}[2]{#1-#2\xspace}

\newenvironment{solution}{}{}
\newenvironment{question}{\renewcommand{\problabel}{Question}\begin{problem}}
{\end{problem}\renewcommand{\problabel}{Problem}}

\SOL{1} % show solution

\makeatletter
\newlength\rpLW
\newlength\rpRW
\newlength\rpgap
\newcommand{\rprcent}{0}
\newcommand{\rplcent}{0}
\define@key{rp}{lwidth}{\setlength{\rpLW}{#1}  \setlength{\rpRW}{\columnwidth-#1-\rpgap}}
\define@key{rp}{rwidth}{\setlength{\rpRW}{#1} \setlength{\rpLW}{\columnwidth-#1-\rpgap}}
 \define@key{rp}{lvert}{\def\LeftVerticalCode{#1}}
\define@key{rp}{rvert}{\def\RightVerticalCode{#1}}
\define@key{rp}{gap}{\setlength{\rpgap}{#1}}
\define@key{rp}{rcent}{\def\rprcent{#1}}
\define@key{rp}{lcent}{\def\rplcent{#1}}
\setkeys{rp}{rwidth=2in, lvert=c, rvert=t, gap=0.25in, rcent=0, lcent=0}
\makeatother

\newcommand\rp[3][]{%
  \setkeys{rp}{#1}
  \noindent \parbox[\LeftVerticalCode]{\rpLW}%
  {\ifthenelse{\equal{\rplcent}{0}}%
  {#2}%
  { \begin{center}#2\end{center} }%
  }%
  \hspace*{\rpgap}%
  \parbox[\RightVerticalCode]{\rpRW}%
  {\ifthenelse{\equal{0}{\rprcent}}%
  {#3}%
  {\begin{center} #3 \end{center}}}%
}

\newcommand{\DD}{\ensuremath{d}}		% differential d without leading space
\newcommand{\dd}{\ensuremath{\,\DD}}	% differential d

\newcommand{\deriv}[3][]{\ensuremath{%
	\ifthenelse{\equal{#1}{}}{\frac{\DD #2}{\DD #3}}
	{\frac{\DD^{#1} #2}{\DD #3^{#1}}}}}

\newcommand{\pd}[3][]{\ensuremath{%
	\ifthenelse{\equal{#1}{}}{\frac{\partial #2}{\partial #3}}
	{\partial #2 / \partial #3}}}

\newcommand{\so}{\ensuremath{\qquad\Longrightarrow\qquad}}
\newcommand{\uth}{\ensuremath{\boldsymbol{\hat{\theta}}}\xspace}
\newcommand{\eref}[2][]{%
	\ifthenelse{\equal{#1}{}}%
	{Eq.~(\ref{eq:#2})}%
	{Equation~(\ref{eq:#2})}\xspace}
\newcommand{\cross}{\ensuremath{\times}\xspace}
\newcommand{\into}{\ensuremath{\hat{\otimes}}}	% direction symbols
\newcommand{\outof}{\ensuremath{\hat{\odot}}}
\newcommand{\ds}{\displaystyle}

\newcommand{\xdd}{\ensuremath{\ddot{x}}\xspace}
\newcommand{\thdd}{\ensuremath{\ddot{\theta}}\xspace} % Download this file from physics.hmc.edu/motion/ page
\usepackage{tikz}

\usetikzlibrary{arrows,calc}
                	\tikzset{%
                         every picture/.style={>=stealth'},
                                        vel/.style={->,line width=2pt,color=DarkBlue},
                  force/.style={line width=1.5pt,color=blue,->},
                  coord/.style={color=green!40!black,|->},
                  accel/.style={->,line width=3pt,color=gray},
                  photon/.style={line width=1.5pt,color=DarkRed,decorate,decoration={snake,post length=0.1in}},
                  spring/.style={decorate,decoration={coil,aspect=0.3,segment length=2mm,amplitude=2mm}},
                  traj/.style={dashed, color=gray, line width=1pt}
                }
                
\usepackage{fullpage}
\setlength{\parskip}{6pt}
\setlength{\parindent}{0pt}
\usepackage[margin=1in]{geometry}
\usepackage{graphicx}
\usepackage{enumerate}
\usepackage{marvosym}
\usepackage{amssymb}
\usepackage{wasysym}
\usepackage{gensymb}
\usepackage{mathrsfs}
\usepackage{scrextend}
\usepackage{mathtools}
\usepackage{pgfplots}
\usepackage{xspace}
\usepackage[colorlinks]{hyperref}

% --- style --- %
\renewcommand{\labelenumi}{{ (\alph{enumi})}}
\newcommand{\sand}{\quad \mbox{ and } \quad}
%\newcommand{\ds}{\displaystyle}
\allowdisplaybreaks

% --- making \xi look less awful --- %
\DeclareSymbolFont{CMletters}{OML}{cmm}{m}{it}
\DeclareMathSymbol{\xi}{\mathord}{CMletters}{"18}

% --- math --- %
\newcommand{\Z}{\mathbb{Z}}
\newcommand{\R}{\mathbb{R}}
\newcommand{\C}{\mathbb{C}}
\newcommand{\Q}{\mathbb{Q}}


\newcommand{\Lt}[1]{\mathcal{L}\crb{#1}}
\newcommand{\ilt}[1]{\mathcal{L}^{-1}\crb{#1}}

\newcommand{\pn}[1]{\left( #1 \right)}
\newcommand{\sqb}[1]{\left[ #1 \right]}
\newcommand{\crb}[1]{\left\{ #1 \right\}}
\newcommand{\lra}[1]{\left\langle #1 \right\rangle}
\newcommand{\magn}[1]{\left\lVert #1 \right\rVert}

\newcommand{\pdr}[2]{\frac{\partial #1}{\partial #2}}
\newcommand{\im}[1]{\text{im}\pn{#1}}
\newcommand{\m}[1]{\Z/#1\Z}

\DeclareMathOperator{\proj}{proj}
\newcommand{\vectorproj}[2][]{\proj_{\VEC{#1}}\VEC{#2}}

\newenvironment{amatrix}[1]{%
  \left(\begin{array}{@{}*{#1}{c}|c@{}}
}{%
  \end{array}\right)
}

\newcommand{\spn}[1]{\text{span}\pn{#1}}

\newcommand*\Heq{\ensuremath{\overset{\kern2pt H}{=}}}

\newcommand{\distil}{\sqrt{1-v^2/c^2}}
\newcommand{\distilf}[1]{\sqrt{1-(#1)^2}}
\newcommand{\lorentz}{\frac{1}{\distil}}
\newcommand{\lorentzf}[1]{\frac{1}{\sqrt{1-(#1)^2}}}

\begin{document}

\noindent{\large Problem Set 1, 10 September 2018\hfill Name: \underline{\hspace{3cm}} ,  Section: \underline{\hspace{5mm}} }
\vspace*{0.25in}


\begin{problem}[(E25.30)]
Two physics students (Mary at \val{52.0}{kg} and John at \val{90.7}{kg}) are \val{28.0}{m} apart. Let each have a 0.01\% imbalance in their
amounts of positive and negative charge, one student being positive and the other negative. Estimate the electrostatic force of attraction between them.
(Hint: Replace the students by spheres of water and use the result of Exercise 29.)
\end{problem}


% Your solution starts here %%%%%%%%%%%%%%%%%%%%%%%%%%%%%%%%%%%%%%%%%%%%%%%%%%
\textbf{Solution:}

% Your solution ends here %%%%%%%%%%%%%%%%%%%%%%%%%%%%%%%%%%%%%%%%%%%%%%%%%%

\clearpage
\begin{problem}[(P25.4(a))*]
Two similar tiny balls of mass $m$ are hung from silk threads of length $L$ and carry equal charges $q$ as in the figure below. Assume that
$\theta$ is so small that $\tan\theta$ can be replaced by approximate equal, $\sin\theta$. (a) To this approximation show that, for equilibrium,
$$
	x = \pn{\frac{q^2L}{2\pi\epsilon_0mg}}^{1/3},
$$
where $x$ is the separation between the balls.
\begin{center}
\includegraphics[scale=0.6]{prob2.png}
\end{center}
\end{problem}


% Your solution starts here %%%%%%%%%%%%%%%%%%%%%%%%%%%%%%%%%%%%%%%%%%%%%%%%%%
\textbf{Solution:}

% Your solution ends here %%%%%%%%%%%%%%%%%%%%%%%%%%%%%%%%%%%%%%%%%%%%%%%%%%

\clearpage

\begin{problem}[(E26.16)]
A thin glass rod is bent into a semicircle of radius $r$. A charge $+q$ is uniformly distributed along the upper half and a charge $-q$ is uniformly
distributed along the lower half, as shown in the figure below. Find the electric field $\VEC{E}$ at $P$, the center of the semicircle.
\begin{center}
\includegraphics[scale=0.75]{prob3.png}
\end{center}
\end{problem}


% Your solution starts here %%%%%%%%%%%%%%%%%%%%%%%%%%%%%%%%%%%%%%%%%%%%%%%%%%
\textbf{Solution:}

% Your solution ends here %%%%%%%%%%%%%%%%%%%%%%%%%%%%%%%%%%%%%%%%%%%%%%%%%%

\clearpage

\begin{problem}[(P25.8)]
Two positive charges $+Q$ are held fixed a distance $d$ apart. A particle of negative $-q$ and mass $m$ is placed mid-way between them, 
then is given a small displacement perpendicular to the line joining them and released. Show that the particle describes simple harmonic motion of period
$$\pn{\epsilon_0m\pi^3d^3/qQ}^{1/2}.$$\\
\textit{Hint:} Remember the small angle approximation for $\sin x$ when $x$ is small, and remember that for simple harmonic oscillations, restoring force is linearly
proportional to displacement, as in $F = -kx$.
\end{problem}


% Your solution starts here %%%%%%%%%%%%%%%%%%%%%%%%%%%%%%%%%%%%%%%%%%%%%%%%%%
\textbf{Solution:}

% Your solution ends here %%%%%%%%%%%%%%%%%%%%%%%%%%%%%%%%%%%%%%%%%%%%%%%%%%

\clearpage

\begin{problem}[(E26.24)]
(a) In the figure below, locate the point (or points) at which the electric field is zero. (b) Sketch qualitatively the field lines.
\begin{center}
\includegraphics[scale=0.8]{prob5.png}
\end{center}
\end{problem}


% Your solution starts here %%%%%%%%%%%%%%%%%%%%%%%%%%%%%%%%%%%%%%%%%%%%%%%%%%
\textbf{Solution:}

% Your solution ends here %%%%%%%%%%%%%%%%%%%%%%%%%%%%%%%%%%%%%%%%%%%%%%%%%%

\clearpage

\begin{problem}[\P (E26.18)]
An insulating rod of length $L$ has charge $-q$ uniformly distributed along its length, as shown in the figure below. (a) What is the linear charge density of the rod? (b) Find the electric
field at point $P$ a distance $a$ from the end of the rod. (c) If $P$ were very far from the rod compared to $L$, the rod would look like a point charge. Show that your answer to (b) reduces to the electric field of a point charge for $a \gg L$.
\begin{center}
\includegraphics[scale=0.75]{prob6.png}
\end{center}
\end{problem}


% Your solution starts here %%%%%%%%%%%%%%%%%%%%%%%%%%%%%%%%%%%%%%%%%%%%%%%%%%
\textbf{Solution:}

% Your solution ends here %%%%%%%%%%%%%%%%%%%%%%%%%%%%%%%%%%%%%%%%%%%%%%%%%%

\clearpage

\end{document}